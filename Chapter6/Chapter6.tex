%=== CHAPTER SIX (6) ===
%=== Conclusion and Recommendations ===

\chapter{Conclusion \& Future Work}
\begin{spacing}{1.5}
\setlength{\parskip}{0.3in}

% A novel method for simulating skin spot changes is proposed, utilizing a physics-based model coupled with expertise in dermatology, to successfully achieve the modeling of facial skin blemishes. Based on this, an efficient system for precise simulation of blemish changes over an extended period is developed, facilitating highly controllable, natural, and authentic adjustments to the appearance of blemishes. Experiments demonstrate that this algorithm is broadly applicable to various skin tones and types of pigmentations. In comparison to learning-based image manipulation algorithms, this method does not require learning pigmentation patterns from large data sets, yet can achieve results that are comparable in quality.

% This method also has limitations. For instance, it requires users to manually select the blemish area rather than being able to predict their locations automatically. Moreover, the user study highlighted that while a significant percentage of the altered images were perceived as realistic, there remains a portion of images where the alterations were detectable. This underscores the need for continuous improvement in the simulation's subtlety and realism, particularly when dealing with high variability in human perception. Additionally, the parameter settings have only been tested and validated on a self-collected dataset, and whether the proposed algorithm can be applied to images captured in the wild with more complex lighting situations requires further verification.

% In conclusion, this research carves new possibilities in the cosmetic industry. With future improvements, this method has the potential to drive innovation and customization in skin care products, meeting the ever-growing demands of consumers. This work is hopeful to provide valuable insights and inspiration for future exploration in the cross-disciplinary field of computer vision and skin science.
\section{Conclusion}
In conclusion, the novel method presented in this research for simulating skin blemish changes marks a significant advancement in the realm of dermatological technology. By utilizing a physics-based model enriched with dermatological expertise, this approach successfully models facial skin blemishes, offering a highly controllable, natural, and authentic simulation of blemish changes over time. This system, demonstrating broad applicability across various skin tones and blemishes, stands out for its ability to achieve quality results without relying on extensive learning from large datasets, a common limitation in many learning-based image manipulation algorithms.

However, the method is not without its challenges. The requirement for manual selection of blemish areas points to a need for automation in future iterations, potentially through the integration of advanced detection algorithms. Additionally, the user study results suggest a necessity for refining the simulation's subtlety to enhance its realism further, particularly in the face of diverse human perceptions. The parameter settings, while effective within the scope of a controlled dataset, warrant additional testing in more variable, real-world scenarios to ensure broader applicability and reliability.

\section{Future Work}
Looking ahead, this research opens up new pathways in the skincare industry, particularly in the development of skincare products. The potential for this method to foster innovation and customization is significant, aligning well with the evolving demands of consumers seeking personalized skincare solutions. Moreover, the insights gained from this research could act as a catalyst for future interdisciplinary explorations, bridging the gap between computer vision and skin science. The development of more sophisticated, user-friendly, and versatile tools for skin blemish simulation and analysis could bring benefits for both consumer experience and dermatological research. As this field continues to evolve, the integration of machine vision, and image processing with dermatological knowledge will likely unveil new horizons in skincare and treatment, potentially transforming both daily skincare routines and professional dermatological practices.

%=== END OF CHAPTER SIX ===
\end{spacing}
    
\newpage
