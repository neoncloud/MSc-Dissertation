%=== CHAPTER SIX (6) ===
%=== Conclusion and Recommendations ===

\chapter{Conclusion}
\begin{spacing}{1.5}
\setlength{\parskip}{0.3in}

A novel method for simulating skin spot changes is proposed, utilizing a physics-based model coupled with expertise in dermatology, to successfully achieve the modeling of facial skin blemishes. Based on this, an efficient system for precise simulation of blemish changes over an extended period is developed, facilitating highly controllable, natural, and authentic adjustments to the appearance of blemishes. Experiments demonstrate that this algorithm is broadly applicable to various skin tones and types of pigmentations. In comparison to learning-based image manipulation algorithms, this method does not require learning pigmentation patterns from large data sets, yet can achieve results that are comparable in quality.

This method also has limitations. For instance, it requires users to manually select the blemish area rather than being able to predict their locations automatically. Additionally, the parameter settings have only been tested and validated on a self-collected dataset, and whether the proposed algorithm can be applied to images captured in the wild with more complex lighting situations requires further verification.

In conclusion, this research carves new possibilities in the cosmetic industry. With future improvements, this method has the potential to drive innovation and customization in skin care products, meeting the ever-growing demands of consumers. This work is hopeful to provide valuable insights and inspiration for future exploration in the cross-disciplinary field of computer vision and skin science.

%=== END OF CHAPTER SIX ===
\end{spacing}
    
\newpage
