%=== FRONT PART ===
%=== ABSTRCT ===
\newpage

\chapter*{\centering Abstract}
\markboth{Abstract}{}
% \vspace{-0.3in}

\begin{spacing}{1.5}
\setlength{\parskip}{0.3in}

\addcontentsline{toc}{chapter}{Abstract}

Facial blemishes, such as acne and pigmentation, significantly impact skin health and play a crucial role in the perceptions of age and beauty across various age groups and skin tones. The lack of robust simulation techniques to assess changes in facial blemishes present a notable challenge to the skincare industry in studying the efficacy of skin care product and demonstrating it to consumers. To bridge this critical gap, we propose an efficient framework for simulating changes in skin blemishes. Our method is based on prior knowledge that links the appearance of acne and pigmentation to melanin and heamoglobin chromophores under the skin surface. Our novel framework models the spatial distributions of chromophores based on the optical scattering properties of the skin. A unique feature of our method is the precise and stable manipulation of parameters of chromophore distributions, thereby enabling control over the appearance of skin blemishes. We validate our proposed method using a comprehensive dataset containing temporal data on long-term skin blemish changes. Our results show that our method achieves highly realistic simulations. Furthermore, a visual perception study has also demonstrated the authenticity and quality of our simulation method.

\par
\textbf{Keywords:} Facial Image Retouching, Computer Vision, Skin Image.
\end{spacing}
\newpage
%=== END OF ABSTRACT ===
