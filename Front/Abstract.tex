%=== FRONT PART ===
%=== ABSTRCT ===
\newpage

\chapter*{\centering Abstract}
\markboth{Abstract}{}
% \vspace{-0.3in}

\begin{spacing}{1.5}
\setlength{\parskip}{0.3in}

\addcontentsline{toc}{chapter}{Abstract}

Facial image retouching aims to remove facial blemishes from images, such as acne and pigmentation, and still retain textures and details. Nevertheless, existing methods just completely remove the blemishes but focus little on realism of the intermediate process, limiting their use more to beautifying facial images on social media rather than being effective tools for simulating changes in facial pigmentation and acne to demonstrate the efficacy of skincare product to consumers or to assess the skincare product development. To bridge this critical gap, an efficient framework is proposed for simulating changes in skin blemishes. This method is based on prior knowledge that links the appearance of acne and pigmentation to melanin and heamoglobin chromophores under the skin surface. This novel framework models the spatial distributions of chromophores under the optical scattering properties of the skin. A unique feature of this framework is the precise and stable manipulation of parameters of chromophore distributions, thereby enabling control over the appearance of skin blemishes. The proposed framework is validated using a comprehensive dataset containing temporal data on long-term skin blemish changes. Experiment results show that this framework achieves highly realistic simulations. Furthermore, a visual perception study has also demonstrated the authenticity and quality of the proposed simulation method. Moreover, a user-friendly graphic interface is implemented to manipulate the facial blemishes interactively, achieving realistic and gradual blemishes retouching with an easy access.

\par
\textbf{Keywords:} Facial Image Retouching, Computer Vision, Skin Image.
\end{spacing}
\newpage
%=== END OF ABSTRACT ===
