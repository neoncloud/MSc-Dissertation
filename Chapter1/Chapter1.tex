%=== CHAPTER ONE (1) ===
%=== INTRODUCTION ===

\chapter{Introduction}
\begin{spacing}{1.5}
\setlength{\parskip}{0.3in}

\section{Background}

Facial appearance plays a pivotal role in an individual's self-confidence and perception of health and beauty. Among the various factors that contribute to facial aesthetics, the presence of facial blemishes such as acne and pigmentation is critical. These imperfections not only affect physical appearance but also have significant psychological and emotional consequences. Consumers across different age groups and skin tones use various skin treatments such as topical skincare products, chemical peeling, laser treatment, etc. to treat these blemishes and improve their skin appearance\cite{doi:10.2352/EI.2023.35.7.IMAGE-276}.
% TODO: Add supporting material

The relentless pursuit of beauty has catalyzed the growth of an expansive skincare market. The increasing demand for aesthetic improvement from consumers has driven skincare manufacturers to seek intuitive tools that can vividly demonstrate the long-term benefits of their products. Such a tool would enable consumers to visualize and trust the efficacy of skincare products without the need for extensive real-image data collection. Additionally, it would allow manufacturers to gather user feedback objectively, measure the therapeutic effectiveness of their products, and refine their offerings to better meet consumer needs. This pursuit aligns with a broader trend where visual representation and consumer trust are paramount, and where the market's ability to provide clear evidence of product benefits can significantly influence purchasing decisions.

\section{Motivation}

However, consumers often have limited ability to assess the efficacy of skincare treatments designed to address blemishes before starting a treatment\cite{doi:10.2352/EI.2023.35.7.IMAGE-276}. There is a dearth of effective models that can convey the visually appealing changes of blemish evolution to consumers, making the choice of the right skincare product to be more a trial-and-error process, during which individuals may need to use the product for a period of time to see the skin improvement. With robust blemish simulation tools, this uncertainty could be addressed. Furthermore, these tools would enable researchers and product developers to quantitatively assess how different formulations and ingredients impact the appearance of facial blemishes over time.

Although some algorithms currently focused on facial image retouching and achieving the goal of removing skin blemishes while retaining the skin texture details, they are usually designed to beautify facial images to be posted to social media\cite{xieBlemishawareProgressiveFace2023, linExemplarbasedFreckleRetouching2019, shafaeiAutoRetouchAutomaticProfessional2021}. These algorithms, adept at completely removing blemishes, lack the capability to represent the gradual and nuanced effects of blemish recovering process due to skincare treatments. This is partially due to the complex physiological and optical properties of skin, presenting a significant challenge in developing a model that accurately measures and simulates the appearance and evolution of skin blemishes.

Recent deep generative models, such as Generative Adversarial Networks\cite{goodfellowGenerativeAdversarialNetworks2014} (GANs) and diffusion models\cite{DBLP:conf/nips/HoJA20,rombach2021highresolution}, have made prominent progress in image generation and manipulation, there are two main challenges in applying such methods in the blemish simulation task. The first challenge is the collection and labelling of a large amount of high-fidelity skin data. It is well known that deep generative models are data-starving. Lacking a large amount of high-quality training data leads to unrealistic output, artifacts, or even modal collapse. The second challenge is the difficulty of defining the distributions and variations of skin blemishes. The deep generative model is intrinsically conducting distribution mapping on images. While it is easy to define distributions in the task of style transfer\cite{DBLP:conf/iclr/DumoulinSK17,DBLP:conf/iccv/ZhuPIE17, DBLP:journals/corr/GatysEB15a} according to image styles, such as art painting and sketching, the appearance status of acne and pigmentation, it improves or worsens, is hard to classify due to the lack of properly labelled data. Thus, the output of a deep neural network could have entangled features, creating an unacceptable perception to users.

%  Although r
%Moreover, the output of deep neural network could be difficult to be controlled, features and appearance of spots may be entangled with each other, making an unacceptable perception to users.
%Physics-Based modelling for Precise Control

\section{Objectives and Specifications}

Motivated by the above discussion, parametric techniques are sought to achieve lightweight and stable simulation and a physics-based modelling method for simulating skin acne and pigmentation changes is proposed. It is rooted in the domain knowledge of skin physiology and optics research that the appearance of facial skin blemishes: acne, and pigmentations, are related to subcutaneous melanin and haemoglobin chromophores\cite{tsumuraImagebasedSkinColor}. And the unique appearance of skin blemishes is rendered by scattering properties of skin tissue\cite{10.5555/2383894.2383946}. First, a color space transformation is conducted to extract chromophore components from sRGB images. Based on the skin scattering properties, the relative spatial distributions for each component are then constructed by Sum-of-Gaussians. This enables the proposed method to perform realistic blemish simulation, precisely modifying the appearance of facial blemishes by tuning the parameters of the fitted model.

% Validation and User Perception Study
To validate that the proposed method can achieve realistic results, a visual comparison study was first conducted to compare simulated images and the ground-truth images from a self-collected dataset, where temporal data reflects long-term skin blemish changes. The results demonstrated that a high degree of realism is achieved by the simulations when compared to ground-truth images. Secondly, the proposed method was compared with some current generalized image editing/generation algorithms or software. Compared to these methods, the proposed method achieved natural-looking editing of skin blemishes with lower FID scores while producing fewer artifacts than deep learning methods. Furthermore, a visual perception study was conducted to quantitatively assess the discernment abilities of individuals between simulated images and authentic ones. The findings demonstrated that the approach generates realistic representations of skin blemish changes.


\section{Major contribution of the Dissertation}

This innovative approach not only addresses a pressing need in the skincare industry but also promises to impact the product development processes. By providing a reliable tool for simulating and assessing skin blemish changes, the proposed methodology equips skincare researchers and developers with the means to create more effective and targeted products. Moreover, it empowers consumers to make informed choices regarding their skincare routines. The contribution of this work is summarized as follows:
\begin{itemize}
    \item The problem of blemish change simulation is identified, utilizing a physics-based modelling approach to approximate the physiological and optical properties of the skin. By adjusting the parameters of the fitted model, the appearance of skin blemishes can be modified, thereby achieving blemish change simulation.
    \item The visualization results and perception study demonstrate that the proposed method achieves a realistic skin blemish change simulation, suggesting that the physics-based modelling technique is a robust tool for skin science research.
    \item The research provides a new use case for the application of computer vision algorithms in the skincare and cosmetic industry, offering promising prospects in product development and serving as a powerful tool to visualize blemish changes, thus giving customers a more intuitive display of product effects and promoting sales. It also extends existing blemish retouching methods by offering a gradual and natural modification approach, which can be integrated into current face image beauty applications to enhance their realism.
          % and people hardly discriminate the simulated face images from ground-truth images.
\end{itemize}

\section{Organisation of the Dissertation}

This dissertation is organized as follows:
\begin{itemize}
    \item \textbf{Chapter 2} delves into existing literature and technological advancements, providing a thorough background on skin modelling techniques. It first reviewed the definition and physiological features of skin blemishes. Then, it briefly discusses the skin optical models commonly used in realistic skin rendering, whose assumptions and simplified models form the basis for the method proposed in this thesis. Finally, it reviews current face image editing and retouching approaches to identify gaps from multiple angles: the lack of gradual and realistic evolution, the need for face blemish dataset with diverse skin tones, and the limitations in capturing the nuanced and progressive changes in skin appearance.
    % reviews literature on the main research subject of this thesis: facial skin and facial skin images. After briefly introducing the causes of skin blemishes, this chapter focuses on current skin image modeling, rendering, and editing methods.
    \item \textbf{Chapter 3} presents the core methodology of the research, introducing proposed novel method from the principles of skin optic and skin physiology to implementation. It starts with the Beer-Lambert law to model the relationship between chromophore concentration and image pixel values, and proposes to decompose RGB color space into chromophore color space, and then to separate the facial skin image into base and blemish layer. Then, it discusses how to use the Sum-of-Gaussian to approximate the distribution of chromophore, as well as how to modify the input image to achieve a gradual and natural manipulation of skin blemishes. Finally, it presents the algorithm optimization design, pseudocode implementation, and graphic user interface design.
    %  It discussed how to convert the principles of skin optic and skin physiology to implementation, proposes to decompose RGB color space into chromophore color space, and then to separate the facial skin image into base and blemish layer. Finally, an efficient Sum-of-Gaussian approximation is proposed to model the relative concentration of chromophores. It also introduces the workflow of the entire simulation system and the graphic user interface.
    \item \textbf{Chapter 4} thoroughly assesses the simulation algorithm's authenticity and natural appeal through various tests, encompassing both objective and subjective reviews. This chapter details experimental design aspects like parameter choices, data collections, and methodologies. It also highlighting the dataset's comprehensiveness in terms of image resolution, diversity of skin types, and the range of blemishes represented, making it apt for testing the proposed model's efficacy.
    \item \textbf{Chapter 5} articulates the results and insights gained from the conducted experiments. It describes both objective and subjective tests carried out to validate the model. Objective evaluations include comparisons with existing image retouching or generation methods using objective metrics, while subjective tests involve human panel assessments to gauge the naturalness of the simulated skin changes.
    \item \textbf{Chapter 6} summarizes the key techniques of this method, highlighting its advantages. It also discusses its limitations and shortcomings. Finally, it explores the proposed method's application prospects and directions for improvement.
\end{itemize}

\end{spacing}
%=== END OF CHAPTER ONE ===
\newpage


