%=== CHAPTER TWO (2) ===
%=== Literature Review ===

\chapter{Literature Review}
\begin{spacing}{1.5}
\setlength{\parskip}{0.3in}

% \section{Overview}
In this chapter, the definition and physiological features of skin pigmentation are first reviewed. Then, the field of computer graphics is examined, discussing how to model skin and pigmentation to achieve realistic skin image rendering. Finally, attention is turned to the field of computer vision, where state-of-the-art image modeling and editing methods are reviewed, assessing the degree of fit and gaps between the goals of this task and existing methods.

\section{Skin Chromophores \& Pigmentation}
\begin{figure}
    \centering
    \includegraphics[width=0.9\columnwidth]{Chapter2/HM_abs.png}
    \caption{Spectral absorption coefficients of skin chromophore. This work focused on modelling heamoglobin and melanin distribution of skin pigmentation. Image taken from\cite{10.5555/2383894.2383946}}.
    \label{fig:hm_abs}
\end{figure}

What gives human skin its diverse colors? When light is transmitted into the skin, energy of different wavelengths is selectively absorbed by the chromophores, scattered by the skin tissues and then observed in unique colors. The color of human skin and skin pigmentations is primarily influenced by several key chromophores, namely \textit{Melanin}, \textit{Hemoglobin}, \textit{Carotene}, and \textit{Bilirubin}. These pigments, each with unique optical properties, contribute to the skin's overall coloration and appearance:

\begin{itemize}
    \item \textbf{Hemoglobin} Found in red blood cells, Hemoglobin gives blood its red color. The optical properties of Hemoglobin vary between its two forms: oxy-Hemoglobin (oxygen-rich) and deoxy-Hemoglobin (oxygen-poor). These forms have distinct absorption peaks in the visible spectrum, contributing to the reddish undertones of skin.
    \item \textbf{Melanin} Rather than being a singular entity, Melanin is a composite of various polymers, exhibiting a spectrum of shades ranging from pale yellow to deep brown or black. The lighter variants of melanin predominantly consist of \textit{pheomelanin}, whereas \textit{eumelanin} typically constitutes the darker forms of melanin\cite{alalufEthnicVariationMelanin2002a}. This is the primary determinant of skin color\cite{doiSpectralEstimationHuman2003}, providing shades from light to dark. Melanin absorbs across a broad range of the visible spectrum but particularly in the ultraviolet (UV) region\cite{ANDERSON198113}. This absorption is crucial as it protects the skin from UV radiation damage. 
    \item \textbf{Carotene and Bilirubin} These pigments impart a yellowish hue to the skin. They absorb light in the blue region of the spectrum, which complements the reds of Hemoglobin and the browns of melanin, contributing to the overall skin tone\cite{ANDERSON198113}.
\end{itemize}


In this work, mainly hemoglobin and melanin in the skin are considered. For the other chromophores and their appearance, they are used as residual terms. In Figure\ref{fig:hm_abs}, the spectral absorption coefficients of these two key chromophores are shown. Both types of hemoglobin have high absorption coefficients from 400nm to 450nm and from 520nm to 600nm, giving the skin a pink color appearance. Melanin, on the other hand, absorbs UV and blue-violet more strongly, giving the skin a brown to black appearance.

The formation of skin pigmentations, such as brown spots or red spots, is often associated with an overproduction or uneven distribution of skin chromophores. These pigmentations can result from various factors, including genetic predisposition, hormonal changes, sun exposure, and aging. In response to UV radiation, Melanocytes (melanin-producing cells) increase their production of melanin as a protective mechanism, which can lead to localized darkening of the skin.

\section{Skin Modeling \& Rendering Techniques}

\begin{figure}[t]
    \centering
    \includegraphics[width=0.9\columnwidth]{Chapter2/skin_model2.pdf}
    \caption{Layered skin model. A portion of the incident light undergoes specular reflection, revealed as a skin texture layer. The other part transmits into and is scattered by the Epidermis and Dermis. Melanin and haemoglobin, which are distributed in these two layers, absorb specific wavelengths of light, rendering the skin's characteristic color.}
    \label{fig:skin_model}
\end{figure}

Modeling skin as a layered, semi-transparent material has become common practice in studying the optical properties of skin and realistic skin rendering\cite{10.5555/2383894.2383946}. In this model, the interactions of light with the skin can be thought of as combinations of the following:

\begin{enumerate}
    \item \textbf{Specular reflection:} Light reflection from the surface, caused by oils, water, and stratum corneum of the skin. It captures the surface texture of the skin, such as fine grooves and textures. The proposed method preserves these details unaltered.
    \item \textbf{Subsurface scattering and absorption:} Physiologically, skin is semi-transparent\cite{Igarashi2005TheAO}. Skin constituents such as extra-cellular matrix cause random deflections of incoming light rays, some of which are reflected back to the surface and are observed. This phenomenon is called subsurface scattering. In addition, the chromophore components present in the epidermis and dermis layers, such as melanin and heamoglobin, selectively absorb light propagating in the skin, thus rendering the unique hue of human skin. When chromophore is locally accumulated, it will render a blemish where the color is different from the surrounding skin\cite{ANDERSON198113}. The method emphasizes this unique optical phenomenon to achieve realistic pigmentation modelling and editing.
    \item \textbf{Transmission:} When the light is very strong and shines on thin tissue (such as the ears or fingers under strong light), a unique transmission appearance can be observed against the light source. For blemish change modelling, this aspect is disregarded.
\end{enumerate}

Despite the multilayer skin model describing the unique appearance resulting from skin optical properties well and conforming to the physiological structure of real skin, rendering realistic skin images on a computer has been challenging. 

Thanks to advancements in modern graphics hardware and developments in computer graphics, realistic skin rendering can now be achieved\cite{10.1145/1198555.1198593, 2015ExtendingTD, JIMENEZ2015_CGF}. The key lies in achieving an accurate and efficient simulation of the subsurface scattering behavior of the skin. Although ray tracing and path tracing\cite{wrenninge2017path, chiang2016practical} are regarded as some of the most realistic approximations for the behaviors of light rays, these methods often require massive computations and can be difficult to apply to real-time scenarios, so approximate fast algorithms become the primary consideration. Jensen et al.\cite{10.1145/3596711.3596747} proposed the Bidirectional Surface Scattering Reflectance Distribution Function (BSSRDF) to approximate the light transmission function. Based on their observations and assumptions, in highly scattering media, light scattering tends to be isotropic, so the scattering distribution is only related to the distance from the incident point. Based on this assumption, Eugene et al.\cite{d2007efficient} proposed using a diffusion profile to describe this scattering distribution, thereby achieving efficient and realistic skin rendering. However, it is still challenging to accurately simulate the scattering of the multilayer skin model. Fortunately, Jensen et al.\cite{10.1145/1073204.1073308} pointed out that using the sum of 4 or more Gaussian functions to approximate the diffusion profile of the multi-layer skin model has been proven to be very effective in practice. Moreover, they calculated a set of well-fitted parameters and successfully simulated the diffusion distribution of the multi-layer skin model in the RGB domain.

These methods have inspired the work to take into account the optical properties of the skin in the algorithm, thus achieving realistic blemish simulation.

\section{Controllable Facial Image Editing}

\subsection{Objectives and Definitions}

Learning-based methods aim to learn a projection from latent noise to pixels\cite{goodfellowGenerativeAdversarialNetworks2014,DBLP:conf/nips/HoJA20,DBLP:journals/corr/KingmaW13}. Once successfully trained, control over the generated image can be achieved by editing in their latent spaces\cite{DBLP:journals/corr/abs-1812-04948, DBLP:journals/corr/abs-1907-10786}. Additionally, achieving precise and controllable latent editing requires either encoding control parameters into the input noise, modelled as conditional generation\cite{isolaImagetoimageTranslationConditional2017}, or injecting controls into the forward pipeline, such as Low-Rank Adaption(LoRA)\cite{2021arXiv210609685H} or ControlNet\cite{2023arXiv230205543Z}, etc. These methods all require calibrated and labelled data with model fine-tuning to achieve accurate editing.

This physics-based modelling approach simulates the optical properties and physiological characteristics of the skin to model the relative distribution of localized skin chromophores. This is achieved through fitting the Sum-of-Gaussians. This method allows skin-agnostic control over the shape, color, and size of local skin blemishes to simulate their degradation or deterioration process after fitting. Without extensive training data, this method is comparable in effect to deep learning models, with strong interpretability.

\subsection{Dataset and Stability}

Learning-based approaches are highly dependent on dataset quality. On small datasets, deep neural networks are often prone to over-fitting, showing similar generation patterns or binding certain features to another (e.g., binding specific skin tone to a gender, or certain age range). Additionally, if there are not enough samples reflecting continuous changes in the same subject, it becomes challenging for the model to learn a trajectory that fits reality.

On one hand, recent high-resolution portrait datasets\cite{DBLP:journals/corr/abs-1812-04948} have been proposed, facilitating deep learning models to achieve great success in face image generation. However, they mostly focus on coarse, large-scale features (such as face shape, hairstyle, expression, etc.). On the other hand, datasets for skin texture rendering\cite{Bai_2023_CVPR} have been proposed. But they generally contain “flawless” skin with few real skin texture samples reflecting skin diseases or defects, and there are no corresponding annotations. To the authors' knowledge, there is currently no dataset specifically dedicated to skin blemish generation or editing.


\subsection{Controllability}
It is believed that image content editing methods can be broadly categorized into three classes: “Pixel Space,” “Latent Space,” and “Parameter Space.”
\begin{itemize}
    \item \textbf{Pixel Space}: Methods such as inpainting algorithms use neighboring or most similar pixels to fill blemish positions\cite{doi:10.1080/10867651.2004.10487596, bertalmio2001navier}. Then, they blend between the original and modified image (alpha blending). Although this can simply and directly control pigmentation intensity through the alpha parameter, the adjustment trajectory does not conform to reality, resulting in unnatural editing traces. These methods cannot achieve diverse modifications, like controlling melanin unchanged while only modifying haemoglobin concentration.

    \item \textbf{Latent Space}: Latent space editing can achieve smooth and continuous content editing or style transfer, but the trajectory is unpredictable and entangled. Although decoupling features for isolated modification is feasible, it requires constraints during the learning session. These constraints are hard to define manually, while precise feature control relies on extensive annotated data.

    \item \textbf{Parameter Space}: This blemish simulation/editing based on tuning fitted pigmentation model parameters allows free and independent adjustments to the blemish's appearance, including color, position, shape, and size, without altering skin details. Experimental and survey data confirm that this algorithm's pigmentation modifications align with general human perception, yielding natural transformation.
\end{itemize}

%=== END OF CHAPTER TWO ===
\end{spacing}
\newpage
